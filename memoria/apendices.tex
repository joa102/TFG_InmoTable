
\addtocontents{toc}{ \vspace{5mm} \hrule \par}


\chapter {Contenido de una entrada en .bib}
\begin{table} 
	\begin{center}
	\begin{tabular}{l p{10cm} }
		\textbf{Nombre campo} & \textbf{Descripción} \\ \hline

address&Usualmente la dirección de la editorial\\
author&Nombre(s) del (de los) autor(es)\\
title&Título del libro\\
chapter&El número de un capítulo (o sección, etc)\\
edition&La edición de un libro, por ejemplo,segunda\\
editor&Nombre(s) del (de los) editor(es)\\
howpublished&Forma en que fue publicada la obra\\
institution&Institución responsable de un informe técnico\\
journal&Nombre del periódico o revista\\
key&Empleado para la alfabetización, referencias cruzadas y para crear una clave cuando la información del autor no está disponible. No debe confundirse con la etiqueta usada en el cite y que debe colocarse al inicio de la entrada\\
month&El mes de publicación o, para un trabajo inédito, en el que fue escrito\\
note&Cualquier información adicional que pueda ayudar al lector\\
number&El número del periódico, la revista, el informe técnico o del trabajo en una serie\\
organization&La organización responsable de una conferencia o que publica un manual\\
pages&Números de páginas\\
publisher&El nombre de la editorial. No debe confundirse con el editor\\
school&Nombre de la escuela donde fue escrita una tesis\\
series&El nombre de una serie o conjunto de libros\\
title&El título del trabajo\\
type&El tipo de un informe técnico\\
volume&El volumen de un periódico o una revista, o de algún libro que conste de volúmenes\\
year&El año de publicación. Para un trabajo inédito, el año en que fue escrito. Generalmente debe consistir de cuatro dígitos, por ejemplo 200\\
         \hline
	\end{tabular}
	\end{center}
	\caption{Campos de una entrada en archivo de bibliografía .bib}
\end{table}



  \chapter{Tipos de referencias}
  \label{sec:bib}
    \begin{table} 
	\begin{center}
	\begin{tabular}{l p{10cm} }
	 \hline
   article& Un artículo de un periódico o revista\\
  
    book& Un libro con una editorial que se indica en forma explícita. Los campos requeridos en este caso son author (autor), editor, title (título), publisher (editorial) y year (año)\\
  
    booklet& Una obra que está impresa y encuadernada, pero sin una editorial o institución patrocinadora\\
  
    conference& Lo mismo que inproceedings, incluido para compatibilidad con el lenguaje de marcación Scribe\\
  
    inbook& Una parte de un libro, que puede ser un capítulo (o sección) o un rango de páginas\\
  
    incollection& Una parte de un libro que tiene su propio título\\
  
    inproceedings& Un artículo en las actas de sesiones (proceedings) de una conferencia\\
  
    manual& Documentación técnica\\
  
    mastersthesis& Una tesis de maestría (Master thesis) o proyecto fin de carrera\\
  
    misc& Para uso cuando los demás tipos no corresponden\\
  
    phdthesis& Una tesis de doctorado (Ph D thesis)\\
  
    proceedings& Las actas de sesiones de una conferencia\\
  
    techreport& Un reporte publicado por una escuela u otra institución, usualmente numerado dentro de una serie\\
  
    unpublished& Un documento que tiene un autor y título, pero que no fue formalmente publicado\\
          \hline
	\end{tabular}
	\end{center}
	\caption{Tipos posibles de elementos en bibligrafía con natbib}
\end{table}

 
  \chapter *{Artefactos software}



