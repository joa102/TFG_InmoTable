
 \chapter{Apartado BIBLIOGRAFIA}
 Este apartado es de vital importancia para tener éxito en la consecución del trabajo fin de grado. Los tribunales suelen prestar gran atención a este apartado, probablemente por su formación académica. 
 
 Es todo un mundo el de los estilos bibliográficos y la realización de las citas, se puede encontrar un tutorial en:  \url{http://ci2.ual.es/comunicar-la-informacion/citas-y-referencias-bibliograficas/}. No obstante en las siguientes secciones se indicarán las cuestiones básicas para iniciarse en el tema pero restringuiendose a la utilización de la presente plantilla. La gran ventaja de latex es que cada elemento en las referencias   \lstinline[language=enparrafo]!\bibitem! es considerado como un elemento de una base de datos, de forma que es el compilador el que reordena o coloca la cita de acuerdo con el estilo definido.

 \section{Como citar}
A lo largo del texto de la memoria se incluirán citas a trabajos de otros autores. La inclusión de la cita tiene que ser justo después del texto donde se propone algo relativo al trabajo citado. Para citar, como para todo el \LaTeX{} existen numerosos paquetes, los dos mas conocidos son  biblatex y   natbib . 

Además dada forma de citar y de incorporar items en el apartado referencias lleva asociado un estilo de bibliografía    \lstinline[language=enparrafo]!\bibliographystyle{XXX}!, que aun abre mucho mas el abanico de opciones. 

Es recomendable consultar las ayudas de los paquetes \url{https://www.overleaf.com/learn/latex/Biblatex_bibliography_styles} y \url{https://www.overleaf.com/learn/latex/Bibliography_management_with_natbib}.

Lo habitual es utilizar la orden      \lstinline[language=enparrafo]!\cite{xxx}!, o   \lstinline[language=enparrafo]!\citep(xxx)!  en la ubicación en el texto donde se quiere colocar la cita, el compilador se encargá de procesarla según el estilo.

 \section {Apartado referencias}
  Existen dos formas de incorporar las referencias, directamente en el entorno   \lstinline[language=enparrafo]!\begin{bibliography}! y utilizando las ordenes del lenguaje o bien utilizando un archivo .bib para guardar esas referencias. Se generará un auxiliar .bbl que automáticamente se incorpora en el documento.
  
  Este archivo auxiliar es el que tiene directamente las ordenes del lenguaje.
  
 \begin{verbatim}
  
\begin{thebibliography}{3}

\bibitem{SWEBOK2014}
P.~Bourque and R.~E. Fairley, Eds., \emph{{SWEBOK}: Guide to the Software
  Engineering Body of Knowledge}, version 3.0~ed.\hskip 1em plus 0.5em minus
  0.4em\relax Los Alamitos, CA: IEEE Computer Society, 2014.

\bibitem{kitchenham_2013}
B.~Kitchenham and P.~Brereton, ``A systematic review of systematic review
  process research in software engineering,'' \emph{Information and Software
  Technology}, vol.~55, no.~12, pp. 2049 -- 2075, 2013.

\bibitem{basili1992}
V.~R. Basili, ``Software modeling and measurement: the goal/question/metric
  paradigm. cs-tr-2956, umiacs-tr-92-96,'' University of Maryland, Tech. Rep.,
  1992.
\end{thebibliography}

  \end{verbatim}
 
 
 
 \section{Archivos .bib}
 
 BibTeX, y natbib por extensión, usan un archivo externo en texto plano como base de datos de referencias bibliográficas para generar las bibliografías y sus referencias en documentos con distintos formatos de artículos, libros, tesis, presentaciones, etc. Los nombres de archivos de referencias bibliográficas de BibTeX usualmente terminan o usan la extensión .bib. Los ítems bibliográficos incluidos en un .bib están separados por tipos.
 
 \begin{verbatim}
@inbook{Berander2005,
        address = {Berlin, Heidelberg},
        author = {Berander, Patrik and Andrews, Anneliese},
        booktitle = {Engineering and Managing Software Requirements},
        pages = {69--94},
        publisher = {Springer Berlin Heidelberg},
        title = {{Requirements Prioritization}},
        doi= {https://doi.org/10.1007/3-540-28244-0{\_}4},
        year = {2005}
} 
 \end{verbatim}
 
 Los tipos que son reconocidos por virtualmente todos los estilos de BibTeX se muestran en el anexo \ref{sec:bib}. Pero igual que antes es recomendable recurrir a tutoriales o recursos donde se describan con detalle. Además se recomienda la utilización de algún gestor de referencias bibliográficas tipo Mendeley. \url{https://www.mendeley.com/}